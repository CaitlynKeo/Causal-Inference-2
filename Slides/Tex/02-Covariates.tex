\documentclass{beamer}

% xcolor and define colors -------------------------
\usepackage{xcolor}

% https://www.viget.com/articles/color-contrast/
\definecolor{purple}{HTML}{5601A4}
\definecolor{navy}{HTML}{0D3D56}
\definecolor{ruby}{HTML}{9a2515}
\definecolor{alice}{HTML}{107895}
\definecolor{daisy}{HTML}{EBC944}
\definecolor{coral}{HTML}{F26D21}
\definecolor{kelly}{HTML}{829356}
\definecolor{cranberry}{HTML}{E64173}
\definecolor{jet}{HTML}{131516}
\definecolor{asher}{HTML}{555F61}
\definecolor{slate}{HTML}{314F4F}

% Mixtape Sessions
\definecolor{picton-blue}{HTML}{00b7ff}
\definecolor{violet-red}{HTML}{ff3881}
\definecolor{sun}{HTML}{ffaf18}
\definecolor{electric-violet}{HTML}{871EFF}

% Main theme colors
\definecolor{accent}{HTML}{00b7ff}
\definecolor{accent2}{HTML}{871EFF}
\definecolor{gray100}{HTML}{f3f4f6}
\definecolor{gray800}{HTML}{1F292D}


% Beamer Options -------------------------------------

% Background
\setbeamercolor{background canvas}{bg = white}

% Change text margins
\setbeamersize{text margin left = 15pt, text margin right = 15pt} 

% \alert
\setbeamercolor{alerted text}{fg = accent2}

% Frame title
\setbeamercolor{frametitle}{bg = white, fg = jet}
\setbeamercolor{framesubtitle}{bg = white, fg = accent}
\setbeamerfont{framesubtitle}{size = \small, shape = \itshape}

% Block
\setbeamercolor{block title}{fg = white, bg = accent2}
\setbeamercolor{block body}{fg = gray800, bg = gray100}

% Title page
\setbeamercolor{title}{fg = gray800}
\setbeamercolor{subtitle}{fg = accent}

%% Custom \maketitle and \titlepage
\setbeamertemplate{title page}
{
    %\begin{centering}
        \vspace{20mm}
        {\Large \usebeamerfont{title}\usebeamercolor[fg]{title}\inserttitle}\\
        {\large \itshape \usebeamerfont{subtitle}\usebeamercolor[fg]{subtitle}\insertsubtitle}\\ \vspace{10mm}
        {\insertauthor}\\
        {\color{asher}\small{\insertdate}}\\
    %\end{centering}
}

% Table of Contents
\setbeamercolor{section in toc}{fg = accent!70!jet}
\setbeamercolor{subsection in toc}{fg = jet}

% Button 
\setbeamercolor{button}{bg = accent}

% Remove navigation symbols
\setbeamertemplate{navigation symbols}{}

% Table and Figure captions
\setbeamercolor{caption}{fg=jet!70!white}
\setbeamercolor{caption name}{fg=jet}
\setbeamerfont{caption name}{shape = \itshape}

% Bullet points

%% Fix left-margins
\settowidth{\leftmargini}{\usebeamertemplate{itemize item}}
\addtolength{\leftmargini}{\labelsep}

%% enumerate item color
\setbeamercolor{enumerate item}{fg = accent}
\setbeamerfont{enumerate item}{size = \small}
\setbeamertemplate{enumerate item}{\insertenumlabel.}

%% itemize
\setbeamercolor{itemize item}{fg = accent!70!white}
\setbeamerfont{itemize item}{size = \small}
\setbeamertemplate{itemize item}[circle]

%% right arrow for subitems
\setbeamercolor{itemize subitem}{fg = accent!60!white}
\setbeamerfont{itemize subitem}{size = \small}
\setbeamertemplate{itemize subitem}{$\rightarrow$}

\setbeamertemplate{itemize subsubitem}[square]
\setbeamercolor{itemize subsubitem}{fg = jet}
\setbeamerfont{itemize subsubitem}{size = \small}


% Special characters

\usepackage{collectbox}

\makeatletter
\newcommand{\mybox}{%
    \collectbox{%
        \setlength{\fboxsep}{1pt}%
        \fbox{\BOXCONTENT}%
    }%
}
\makeatother





% Links ----------------------------------------------

\usepackage{hyperref}
\hypersetup{
  colorlinks = true,
  linkcolor = accent2,
  filecolor = accent2,
  urlcolor = accent2,
  citecolor = accent2,
}


% Line spacing --------------------------------------
\usepackage{setspace}
\setstretch{1.1}


% \begin{columns} -----------------------------------
\usepackage{multicol}


% Fonts ---------------------------------------------
% Beamer Option to use custom fonts
\usefonttheme{professionalfonts}

% \usepackage[utopia, smallerops, varg]{newtxmath}
% \usepackage{utopia}
\usepackage[sfdefault,light]{roboto}

% Small adjustments to text kerning
\usepackage{microtype}



% Remove annoying over-full box warnings -----------
\vfuzz2pt 
\hfuzz2pt


% Table of Contents with Sections
\setbeamerfont{myTOC}{series=\bfseries, size=\Large}
\AtBeginSection[]{
        \frame{
            \frametitle{Roadmap}
            \tableofcontents[current]   
        }
    }


% Tables -------------------------------------------
% Tables too big
% \begin{adjustbox}{width = 1.2\textwidth, center}
\usepackage{adjustbox}
\usepackage{array}
\usepackage{threeparttable, booktabs, adjustbox}
    
% Fix \input with tables
% \input fails when \\ is at end of external .tex file
\makeatletter
\let\input\@@input
\makeatother

% Tables too narrow
% \begin{tabularx}{\linewidth}{cols}
% col-types: X - center, L - left, R -right
% Relative scale: >{\hsize=.8\hsize}X/L/R
\usepackage{tabularx}
\newcolumntype{L}{>{\raggedright\arraybackslash}X}
\newcolumntype{R}{>{\raggedleft\arraybackslash}X}
\newcolumntype{C}{>{\centering\arraybackslash}X}

% Figures

% \imageframe{img_name} -----------------------------
% from https://github.com/mattjetwell/cousteau
\newcommand{\imageframe}[1]{%
    \begin{frame}[plain]
        \begin{tikzpicture}[remember picture, overlay]
            \node[at = (current page.center), xshift = 0cm] (cover) {%
                \includegraphics[keepaspectratio, width=\paperwidth, height=\paperheight]{#1}
            };
        \end{tikzpicture}
    \end{frame}%
}

% subfigures
\usepackage{subfigure}


% Highlight slide -----------------------------------
% \begin{transitionframe} Text \end{transitionframe}
% from paulgp's beamer tips
\newenvironment{transitionframe}{
    \setbeamercolor{background canvas}{bg=accent!40!black}
    \begin{frame}\color{accent!10!white}\LARGE\centering
}{
    \end{frame}
}


% Table Highlighting --------------------------------
% Create top-left and bottom-right markets in tabular cells with a unique matching id and these commands will outline those cells
\usepackage[beamer,customcolors]{hf-tikz}
\usetikzlibrary{calc}
\usetikzlibrary{fit,shapes.misc}

% To set the hypothesis highlighting boxes red.
\newcommand\marktopleft[1]{%
    \tikz[overlay,remember picture] 
        \node (marker-#1-a) at (0,1.5ex) {};%
}
\newcommand\markbottomright[1]{%
    \tikz[overlay,remember picture] 
        \node (marker-#1-b) at (0,0) {};%
    \tikz[accent!80!jet, ultra thick, overlay, remember picture, inner sep=4pt]
        \node[draw, rectangle, fit=(marker-#1-a.center) (marker-#1-b.center)] {};%
}

\usepackage{breqn} % Breaks lines

\usepackage{amsmath}
\usepackage{mathtools}

\usepackage{pdfpages} % \includepdf

\usepackage{listings} % R code
\usepackage{verbatim} % verbatim

% Video stuff
\usepackage{media9}

% packages for bibs and cites
\usepackage{natbib}
\usepackage{har2nat}
\newcommand{\possessivecite}[1]{\citeauthor{#1}'s \citeyearpar{#1}}
\usepackage{breakcites}
\usepackage{alltt}

% Setup math operators
\DeclareMathOperator{\E}{E} \DeclareMathOperator{\tr}{tr} \DeclareMathOperator{\se}{se} \DeclareMathOperator{\I}{I} \DeclareMathOperator{\sign}{sign} \DeclareMathOperator{\supp}{supp} \DeclareMathOperator{\plim}{plim}
\DeclareMathOperator*{\dlim}{\mathnormal{d}\mkern2mu-lim}
\newcommand\independent{\protect\mathpalette{\protect\independenT}{\perp}}
   \def\independenT#1#2{\mathrel{\rlap{$#1#2$}\mkern2mu{#1#2}}}
\newcommand*\colvec[1]{\begin{pmatrix}#1\end{pmatrix}}

\newcommand{\myurlshort}[2]{\href{#1}{\textcolor{gray}{\textsf{#2}}}}


\begin{document}

\imageframe{./lecture_includes/mixtape_did_cover.png}


% ---- Content ----



\section{Including Covariates}

\subsection{Inverse probability weighting}

\begin{frame}{Controls}

\begin{itemize}
\item Controls can address omitted variable bias (backdoor criterion), and they can improve precision
\item OLS can accommodate controls, and so we tend to include them so long as they are time varying 
\item But unfortunately, time varying covariates can create problems, especially if the treatment causes the covariates (bad controls, colliders)
\end{itemize}

\end{frame}






\begin{frame}{Inverse probability weighting DiD}

 Abadie (2005) incorporates baseline covariates into the propensity score which are then used as weights to estimate the ATT in a simple 3-step process
	\begin{enumerate}
	\item Calculate each unit's ``after minus before'' (DiD equation)
	\item Estimate the conditional probability of treatment based on baseline covariates (propensity score estimation)
	\item Weight the comparison group's DiD equation with the IPW
	\end{enumerate}

\end{frame}

\begin{frame}{Terms}

\begin{itemize}
\item $t$ is year of treatment which doesn't vary across units (so no differential timing)
\item $Y^1$ and $Y^0$ are potential outcomes (counterfactual versus actual)
\item $D$ is 1 or 0 based on group and time
\item $X_b$ are ``baseline'' covariates \textbf{only} -- they do not vary over time, which means propensity scores are estimated off the $b$ period \textbf{only}
\end{itemize}

\end{frame}

\begin{frame}{Assumptions}

Kind of common for this propensity score literature to only have two assumptions.  But usually the first conditional independence.  Now it is parallel trends because this is DD

\begin{enumerate}
\item Conditional parallel trends $$E[Y^0_t - Y^0_b|D=1,X_b] - E[Y^0_t - Y^0_t | D=0, X_b]$$ (Notice the $b$ subscript.  What is that you think?)
\item Common support $$Pr(D=1)>0; Pr(D=1|X)<1$$ Let's see a picture of common support that I drew.  Apologies it's horrible
\end{enumerate}

\end{frame}

\begin{frame}{Common support}

As we are identifying the ATT, we only need common support with respect to treated units

\bigskip

Your identify assumptions are always with respect to the missing covariates in other words and for the ATT, you are missing $Y^0$ for the treatment group

\bigskip

If we were estimating ATU, we'd be missing $Y^1$ for controls and need common support ($Y$ in treatment for all ranges of control), and for ATE we'd need both

\end{frame}

\begin{frame}{Visualizing propensity score to get common support}

	\begin{figure}
	\includegraphics[scale=0.05]{./lecture_includes/common_support_abadie.png}
	\end{figure}

\end{frame}

\begin{frame}{Definition and estimation}

Defining the ATT parameter of interest
\begin{equation}
ATT=E[Y^1_t - Y^0_t |D_t=1]
\end{equation}

\bigskip
Abadie's estimator
\begin{equation}
E\bigg [ \frac{Y_t - Y_b}{Pr(D_t=1)} \times \frac{D_t - Pr(D=1|X_b)}{1-Pr(D=1|X_b)} \bigg ]
\end{equation}


\end{frame}


\begin{frame}{Propensity scores}

\begin{itemize}
\item It's common to hear people say that we don't know the propensity score; we can only estimate it. Same here -- we approximate it with regressions
\item Paper is titled ``Semi-parametric DiD'' because Abadie imposes structure on the polynomials used to construct the propensity score (``series logit'')
\end{itemize}

\end{frame}



\begin{frame}{Abadie 2005 influence}

	\begin{figure}
	\includegraphics[scale=0.25]{./lecture_includes/abadie_restud_ipw}
	\end{figure}Abadie (2005) is his fourth most cited paper

\end{frame}




\subsection{Double Robust DiD}

\begin{frame}{Doubly Robust Difference-in-differences}

\begin{itemize}
\item DR models control for covariates twice -- once using the propensity score, once using outcomes adjusted by regression -- and are unbiased so long as:
	\begin{itemize}
	\item The regression specification for the outcome is correctly specified
	\item The propensity score specification is correctly specified
	\end{itemize}
\item Sant'Anna and Zhao (2020) incorporated DR into DiD by combining inverse probability weighting and outcome regression into a single DiD model
\item It's in the engine of Callaway and Sant'Anna (2020) that we discuss later so it merits close study
\item One of my favorite lesser known of the new DiD papers
\end{itemize}

\end{frame}

\begin{frame}{Patterns in econometrician reasoning}

\begin{enumerate}
\item Define the target parameter first (as opposed to writing down a regression specification first)
\item Identification (e.g., parallel trends)
\item Estimation
\item Aggregation
\item Inference
\end{enumerate}

\end{frame}


\begin{frame}{Defining the target parameter}

Major part of the new econometrics is to always start with the target parameter and build to it using estimation and identification that ``works''

\bigskip

\begin{eqnarray*}
\delta = E[Y^1_{it} - Y^0_{it} | D_i=1]
\end{eqnarray*}

\end{frame}

\begin{frame}{Identification assumptions I: Data}

Assumption 1: Assume panel data or repeated cross-sectional data

\bigskip

Handling repeated cross-sectional data is possible but assumes stationarity which is a kind of stability assumption, but I'll use panel representation. 

\bigskip

Cross-sections will be potentially violated with changing sample compositions (e.g., the Napster example). 

\end{frame}

\begin{frame}{Identification assumptions II: Modification to parallel trends}

Assumption 2: Conditional parallel trends

\bigskip

Counterfactual trends for the treatment group are the same as the control group for all values of $X$

\begin{eqnarray*}
E[Y_1^0 - Y_0^0 | X, D=1] = E[Y^0_1 - Y^0_0 | X, D=0]
\end{eqnarray*}

\end{frame}

\begin{frame}{Identification assumptions III: Common support}

Assumption 3: Common support

\bigskip

For some $e>0$, the probability of being in the treatment group is greater than $e$ and the probability of being in the treatment group conditional on $X$ is $\leq1-e$. 

\bigskip

Intuition of assumption 3: Called overlap or common support. Means there is at least a small fraction of the population that is treated and that for every value of the covariates $X$ there is at least a small chance that the unit is not treated. It's called common support when it's a propensity score but it's just about the distribution of treatment and control across values of $X$. Very common when dealing with covariate comparisons as otherwise you're extrapolating (curse of dimensionality)

\end{frame}

\begin{frame}{Estimating DD with Assumptions 1-3}

\begin{itemize}
\item Assumptions 1-3 gives us a couple of options of estimating the DiD
\item We can either use the outcome regression (OR) approach of Heckman, et al 1997
\item Or we can use the inverse probability weighting (IPW) approach of Abadie (2005)
\end{itemize}

\end{frame}


\begin{frame}{Heckman, et al. 1997}

	\begin{figure}
	\includegraphics[scale=0.35]{./lecture_includes/petra_restud_1997}
	\end{figure}

\end{frame}



\begin{frame}{Outcome regression}

This is the Heckman, et al. (1997) approach where the outcome evolution is modeled with a regression

\bigskip

\begin{eqnarray*}
\widehat{\delta}^{OR} = \overline{Y}_{1,1} - \bigg [ \overline{Y}_{1,0} + \frac{1}{n^T} \sum_{i|D_i=1} ( \widehat{\mu}_{0,1}(X_i) - \widehat{\mu}_{0,0}(X_i)) \bigg ]
\end{eqnarray*}

where $\overline{Y}$ is the sample average of $Y$ among units in the treatment group at time $t$ and $\widehat{\mu}(X)$ is an estimator of the true, but unknown, $m_{d,t}(X)$ which is by definition equal to $E[Y_t|D=d,X=x]$.

\end{frame}




\begin{frame}{Outcome regression}

\begin{eqnarray*}
\widehat{\delta}^{OR} = \overline{Y}_{1,1} - \bigg [ \overline{Y}_{1,0} + \frac{1}{n^T} \sum_{i|D_i=1} ( \widehat{\mu}_{0,1}(X_i) - \widehat{\mu}_{0,0}(X_i)) \bigg ]
\end{eqnarray*}

\begin{enumerate}
\item Regress changes $\Delta Y$ on $X$ among untreated groups using baseline covariates only
\item Get fitted values of the regression using all $X$ from $D=1$ only.  Average those
\item Calculate change in this fitted $Y$ among treated with the average fitted values
\end{enumerate}

\end{frame}

\begin{frame}{Inverse probability weighting}

This is the Abadie (2005) approach where we use weighting

\begin{eqnarray*}
\widehat{\delta}^{ipw} = \frac{1}{E_N[D]} E \bigg [ \frac{D-\widehat{p}(X)}{1-\widehat{p}(X)} (Y_1-Y_0) \bigg ]
\end{eqnarray*}

where $\widehat{p}(X)$ is an estimator for the true propensity score. Reduces the dimensionality of $X$ into a single scalar.

\end{frame}

\begin{frame}{These models cannot be ranked}

\begin{itemize}
\item Outcome regression needs $\widehat{\mu}(X)$ to be correctly specified, whereas
\item Inverse probability weighting needs $\widehat{p}(X)$ to be correctly specified
\item It's hard to ``rank'' these two in practice with regards to model misspecification because each is inconsistent when their own models are misspecified
\end{itemize}

\end{frame}


\begin{frame}{TWFE}

Consider our earlier TWFE specification:

\begin{eqnarray*}
Y_{it} = \alpha_1  + \alpha_2 T_t + \alpha_3 D_i +  \delta (T_i \times D_t)  + \varepsilon_{it}
\end{eqnarray*}

\bigskip

Just add in covariates then right?

\begin{eqnarray*}
Y_{it} = \alpha_1  + \alpha_2 T_t + \alpha_3 D_i  + \delta (T_i \times D_t) + \theta \cdot X_{it} + \varepsilon_{it}
\end{eqnarray*}

Sure! If you're willing to impose three \emph{more} assumptions

\end{frame}




\begin{frame}{Decomposing TWFE with covariates}

TWFE places restrictions on the DGP. Previous TWFE regression under assumptions 1-3 implies the following:

\bigskip

\begin{eqnarray*}
E[Y^1_1|D=1,X] = \alpha_1 + \alpha_2 + \alpha_3 + \delta + \theta X
\end{eqnarray*}

\bigskip

Conditional parallel trends implies

\small
\begin{eqnarray*}
&&E[Y^0_{1} - Y^0_{0}|D=1,X]= E[Y^0_{1} - Y^0_{0}|D=0,X] \\
&&E[Y^0_{1}|D=1,X] - E[Y^0_{0}|D=1,X]= E[Y^0_{1}|D=0,X] - E[Y^0_{0}|D=0,X] \\
&&E[Y^0_{1}|D=1,X] = E[Y^0_{0}|D=1,X] + E[Y^0_{1}|D=0,X] - E[Y^0_{0}|D=0,X] \\
&&E[Y^0_{1}|D=1,X] = E[Y_{0}|D=1,X] + E[Y_{1}|D=0,X] - E[Y_{0}|D=0,X] \\
\end{eqnarray*}


\end{frame}

\begin{frame}{Switching equation substitution}

Last line from the switching equation. This gives us:

\begin{eqnarray*}
E[Y^0_{1}|D=1,X] = \alpha_1  + \alpha_2 + \alpha_3 + \theta X
\end{eqnarray*}

Now compare this with our earlier $Y^1$ expression

\begin{eqnarray*}
E[Y^1_1|D=1,X] = \alpha_1 + \alpha_2 + \alpha_3 + \delta + \theta X
\end{eqnarray*}

We can define our target parameter, the ATT, now in terms of the fixed effects representation

\end{frame}


\begin{frame}{Collecting terms}

TWFE representation of our conditional expectations of the potential outcomes
\begin{eqnarray*}
&&E[Y^1_1|D=1,X] = \alpha_1 + \alpha_2 + \alpha_3 + \delta + \theta_1 X \\
&&E[Y^0_{1}|D=1,X] = \alpha_1  + \alpha_2 + \alpha_3 + \theta_2 X \\
\end{eqnarray*}

Substitute these into our target parameter

\begin{eqnarray*}
ATT &=& E[Y^1_1|D=1,X]  - E[Y^0_{1}|D=1,X]   \\
&&=(\alpha_1 + \alpha_2 + \alpha_3 + \delta + \theta_1 X) - ( \alpha_1  + \alpha_2 + \alpha_3 + \theta_2 X )\\
&&=\delta + (\theta_1 X - \theta_2 X)
\end{eqnarray*}

\bigskip

What if $\theta_1 X \neq \theta_2 X$?

\end{frame}

\begin{frame}{Assumption 4: Homogeneous treatment effects in X}


TWFE requires homogenous treatment effects in $X$ (i.e., the treatment effect is the same for all $X$)

\bigskip

If $X$ is sex, then effects are the same for males and females.

\bigskip

  If $X$ is continuous, like income, then the effect is the same whether someone makes \$1 or \$1 million.

\end{frame}

\begin{frame}{X-specific trends}

TWFE also places restrictions on covariate trends for the two groups too.  Take conditional expectations of our TWFE equation. 

\begin{eqnarray*}
E[Y_1|D=1] &=& \alpha_1 + \alpha_2 + \alpha_3 + \delta + \theta X_{11} \\
E[Y_0|D=1] &=& \alpha_1 + \alpha_3 + \theta X_{10} \\
E[Y_1|D=0] &=& \alpha_1 + \alpha_2 + \theta X_{01} \\
E[Y_0|D=0] &=& \alpha_1 + \theta X_{00}
\end{eqnarray*}


\end{frame}


\begin{frame}{X-specific trends}

Now take the DiD formula:

\begin{eqnarray*}
\delta^{DD} = &&\bigg ( (\alpha_1 + \alpha_2 + \alpha_3 + \delta + \theta X_{11} ) - (\alpha_1 + \alpha_3 + \theta X_{10} ) \bigg )- \\
&& \bigg ( (\alpha_1 + \alpha_2 + \theta X_{01}) - (\alpha_1 + \theta X_{00}) \bigg )
\end{eqnarray*}

\bigskip

Eliminating terms, we get:

\begin{eqnarray*}
\delta^{DD} = &&\delta + \\
&& (\theta X_{11} - \theta X_{10} ) - (\theta X_{01} - \theta X_{00} )
\end{eqnarray*}

\bigskip

Second line requires that trends in X for treatment group equal trends in X for control group.

\end{frame}


\begin{frame}{Assumption 5 and 6}

We need ``no X-specific trends'' for the treatment group (assumption 5) and comparison group (assumption 6)

\bigskip

\textbf{Intuition}: No X-specific trends means the evolution of potential outcome $Y^0$ is the same regardless of $X$. This would mean you cannot allow rich people to be on a different trend than poor people, for instance.

\bigskip

Without these six, in general TWFE will not identify ATT. 

\end{frame}

\begin{frame}{Why not both?}

\begin{itemize}
\item Let's review the problem.  What if you claim you need $X$ for conditional parallel trends?
\item You have three options:
	\begin{enumerate}
	\item Outcome regression (Heckman, et al. 1997) -- needs Assumptions 1-3
	\item Inverse probability weighting (Abadie 2005) -- needs Assumptions 1-3
	\item TWFE (everybody everywhere all the time) -- needs Assumptions 1-6
	\end{enumerate}
\item Problem is 1 and 2 need the models to be correctly specified
\item Doubly robust combines them to give us insurance; we now get two chances to be wrong, as opposed to just one
\end{itemize}

\end{frame}


\begin{frame}{Double Robust DiD}
\begin{eqnarray*}
\delta^{dr} = E \bigg [ \bigg ( \frac{D}{E[D]} -\frac{ \frac{p(X)(1-D)}{(1-p(X))} }{E \bigg [\frac{p(X)(1-D)}{(1-p(X))} \bigg ]} \bigg  )( \Delta Y - \mu_{0,\Delta}(X)) \bigg ]
\end{eqnarray*}

\begin{eqnarray*}
&&p(x): \text{propensity score model} \\
&& \Delta Y = Y_1 - Y_0 = Y_{post} - Y_{pre} \\
&& \mu_{d,\Delta} = \mu_{d,1}(X) - \mu_{d,0}(X), \text{ where } \mu(X) \text{ is a model for} \\
&& m_{d,t} = E[Y_t|D=d,X=x]
\end{eqnarray*}So that means $\mu_{0,\Delta}$ is just the control group's change in average $Y$ for each $X=x$

\end{frame}

\begin{frame}{Double Robust DiD}

\begin{eqnarray*}
\delta^{dr} = E \bigg [ \bigg ( \frac{D}{E[D]} -\frac{ \frac{p(X)(1-D)}{(1-p(X))} }{E \bigg [\frac{p(X)(1-D)}{(1-p(X))} \bigg ]} \bigg  )( \Delta Y - \mu_{0,\Delta}(X)) \bigg ]
\end{eqnarray*}

Notice how the model controls for $X$: you're weighting the adjusted outcomes using the propensity score

\bigskip

The reason you control for $X$ twice is because you don't know which model is right.  DR DiD frees you from making a choice without making you pay too much for it


\end{frame}

\begin{frame}{Efficiency}

\begin{itemize}
\item Authors exploit all the restrictions implied by the assumptions to construct semiparametric bounds
\item This is where the influence function comes in, which those who have studied the DID code closely may have noticed
\item One of the main results of the paper is that the DR DiD estimator is also DR for inference
\item Let's skip to Monte Carlos
\end{itemize}

\end{frame}

\begin{frame}{Monte Carlo details}

\begin{itemize}
\item Compare DR with TWFE, OR and IPW
\item Sample size is 1,000
\item 10,000 Monte Carlo experiments
\item Propensity score estimated with logit; OR estimated using linear specification
\end{itemize}

\end{frame}



\begin{frame}[plain]

\begin{table}[htbp]\centering
\scriptsize
\caption{Monte Carlo Simulations, DGP1, Both OR and Propensity score correct}
\centering
\begin{threeparttable}
\begin{tabular}{l*{5}{c}}
\toprule
\multicolumn{1}{l}{\textbf{}}&
\multicolumn{1}{c}{\textbf{Bias}}&
\multicolumn{1}{c}{\textbf{RMSE}}&
\multicolumn{1}{c}{\textbf{SE}}&
\multicolumn{1}{c}{\textbf{Coverage}}&
\multicolumn{1}{c}{\textbf{CI length}}\\
\midrule
TWFE & -20.9518 & 21.1227 & 2.5271 & 0.000 & 9.9061 \\
OR & -0.0012 & 0.1005 & 0.1010 & 0.9500 & 0.3960 \\
IPW & 0.0257 & 2.7743 & 2.6636 & 0.9518 & 10.4412 \\
DR & -0.0014 & 0.1059 & 0.1052 & 0.9473 & 0.4124 \\
\bottomrule
\end{tabular}
\end{threeparttable}
\end{table}

\end{frame}


\begin{frame}[plain]
	\begin{figure}
	\includegraphics[scale=0.25]{./lecture_includes/mc_dr_1.png}
	\end{figure}

\end{frame}


\begin{frame}[plain]

\begin{table}[htbp]\centering
\scriptsize
\caption{Monte Carlo Simulations, DGP4, Neither OR and Propensity score correct}
\centering
\begin{threeparttable}
\begin{tabular}{l*{5}{c}}
\toprule
\multicolumn{1}{l}{\textbf{}}&
\multicolumn{1}{c}{\textbf{Bias}}&
\multicolumn{1}{c}{\textbf{RMSE}}&
\multicolumn{1}{c}{\textbf{SE}}&
\multicolumn{1}{c}{\textbf{Coverage}}&
\multicolumn{1}{c}{\textbf{CI length}}\\
\midrule
TWFE & -16.3846 & 16.5383 & 3.6268 & 0.000 & 14.2169 \\
OR & -5.2045 & 5.3641 & 1.2890 & 0.0145 & 5.0531 \\
IPW & -1.0846 & 2.6557 & 2.3746 & 0.9487 & 9.3084 \\
DR & -3.1878 & 3.4544 & 1.2946 & 0.3076 & 5.0749 \\
\bottomrule
\end{tabular}
\end{threeparttable}
\end{table}

\end{frame}

\begin{frame}[plain]
	\begin{figure}
	\includegraphics[scale=0.12]{./lecture_includes/mc_dr_2.png}
	\end{figure}


\end{frame}

\subsection{Lalonde lab}

\begin{frame}{R and Stata Code}

There is code in R and Stata (all DiD estimators are now beautifully arranged at a website hosted by Asjad Naqvi)
\begin{itemize}
\item Stata: \textbf{drdid}
\item R: \textbf{drdid}
\end{itemize}

\bigskip

\url{https://asjadnaqvi.github.io/DiD/docs/01_stata/}

\bigskip

Remember -- it's for 2x2 with covariates (i.e., one treatment group). 

\end{frame}

\begin{frame}{Application using real data}

\begin{itemize}
\item Let's now use a real example with real data and see how well this does
\item Famous paper in AER by Lalonde (1986), an Orley and Card student at Princeton
\item Found that most program evaluation did badly, but let's revisit it with diff-in-diff
\end{itemize}

\end{frame}

\begin{frame}{Description of NSW Job Trainings Program}
	
The National Supported Work Demonstration (NSW), operated by Manpower Demonstration Research Corp in the mid-1970s:
	\begin{itemize}
	\item was a temporary employment program designed to help disadvantaged workers lacking basic job skills move into the labor market by giving them work experience and counseling in a sheltered environment
	\item was also unique in that it \textbf{randomly assigned} qualified applicants to training positions:
		\begin{itemize}
		\item \textbf{Treatment group}: received all the benefits of NSW program
		\item \textbf{Control group}: left to fend for themselves
		\end{itemize}
	\item admitted AFDC females, ex-drug addicts, ex-criminal offenders, and high school dropouts of both sexes
	\end{itemize}
\end{frame}

\begin{frame}{NSW Program}
	
	\begin{itemize}
	\item Treatment group members were:
		\begin{itemize}
		\item guaranteed a job for 9-18 months depending on the target group and site
		\item divided into crews of 3-5 participants who worked together and met frequently with an NSW counselor to discuss grievances and performance
		\item paid for their work
		\end{itemize}
	\item Control group members were randomized so the same
	\item Note: the randomization balanced observables and unobservables across the two arms, thus enabling the estimation of an ATE for the people who self-selected into the program
	\end{itemize}
\end{frame}

\begin{frame}{NSW Program}

\begin{itemize}
	\item Other details about the NSW program:
		\begin{itemize}
		\item \underline{Wages}:  NSW offered the trainees lower wage rates than they would've received on a regular job, but allowed their earnings to increase for satisfactory performance and attendance
		\item \underline{Post-treatment}: after their term expired, they were forced to find regular employment
		\item \underline{Job types}:  varied within sites -- gas station attendant, working at a printer shop -- and males and females were frequently performing different kinds of work
		\end{itemize}
\end{itemize}

\end{frame}
	
\begin{frame}{NSW Data}
	
	\begin{itemize}
	\item \underline{NSW data collection}:
		\begin{itemize}
		\item MDRC collected earnings and demographic information from both treatment and control at baseline and every 9 months thereafter
		\item Conducted up to 4 post-baseline interviews
		\item Different sample sizes from study to study can be confusing, but has simple explanations
		\end{itemize}
	\end{itemize}
\end{frame}
	

\begin{frame}{NSW Data}

\begin{itemize}
	\item \underline{Estimation}:
		\begin{itemize}
		\item NSW was a randomized job trainings program; therefore estimating the average treatment effect is straightforward:
			\begin{eqnarray*}
			\frac{1}{N_t}\sum_{D_i=1}Y_i - \frac{1}{N_c}\sum_{D_i=0}Y_i \approx E[Y^1-Y^0] 
			\end{eqnarray*}in large samples assuming treatment selection is independent of potential outcomes (randomization) -- i.e., $(Y^0,Y^1)\independent{D}$. 
		\end{itemize}
	\item \underline{NSW worked}: Treatment group participants' real earnings post-treatment (1978) was positive and economically meaningful -- $\approx$ \$900 (LaLonde 1986) to \$1,800 (Dehejia and Wahba 2002) depending on the sample used
\end{itemize}

\end{frame}
	
\begin{frame}[plain]
	\begin{center}
	LaLonde, Robert J. (1986). \myurlshort{http://business.baylor.edu/scott_cunningham/teaching/lalonde-1986.pdf}{``Evaluating the Econometric Evaluations of Training Programs with Experimental Data''}. \emph{American Economic Review}. 
	\end{center}
	
\underline{LaLonde's study} was \textbf{not} an evaluation of the NSW program, as that had been done, but rather an evaluation of econometric models done by:
		\begin{itemize}
		\item replacing the experimental NSW control group with non-experimental control group drawn from two nationally representative survey datasets: Current Population Survey (CPS) and Panel Study of Income Dynamics (PSID)
		\item estimating the average effect using non-experimental workers as controls for the NSW trainees 
		\item comparing his non-experimental estimates to the experimental estimates of \$900
		\end{itemize}
\end{frame}

\begin{frame}{LaLonde (1986)}

\begin{itemize}

	\item \underline{LaLonde's conclusion}: available econometric approaches were biased and inconsistent
		\begin{itemize}
		\item His estimates were way off and usually the wrong sign
		\item Conclusion was influential in policy circles and led to greater push for more experimental evaluations
		\end{itemize}

\end{itemize}

\end{frame}

\imageframe{./lecture_includes/lalonde_table5a.png}
\imageframe{./lecture_includes/lalonde_table5b.png}

\begin{frame}[plain,shrink=10]{Imbalanced covariates for experimental and non-experimental samples}

    \begin{center}
		\begin{table}
		\begin{tabular}{lcccccc}
		\hline \hline
		\multicolumn{3}{c}{}&
		\multicolumn{1}{c}{CPS}&
		\multicolumn{1}{c}{NSW}\\
		
		\multicolumn{1}{c}{}&
		\multicolumn{2}{c}{All} &
		\multicolumn{1}{c}{Controls} &
		\multicolumn{1}{c}{Trainees} \\

		\multicolumn{3}{c}{}&
		\multicolumn{1}{c}{$N_c=15,992$}&
		\multicolumn{1}{c}{$N_t=297$}&
		\multicolumn{1}{c}{}&
		\multicolumn{1}{c}{}\\

		\multicolumn{1}{l}{covariate}&
		\multicolumn{1}{c}{mean}&
		\multicolumn{1}{c}{(s.d.)}&
		\multicolumn{1}{c}{mean}&
		\multicolumn{1}{c}{mean}&
		\multicolumn{1}{c}{t-stat}&
		\multicolumn{1}{c}{diff}\\
		\hline
Black    & 0.09 & 0.28 & 0.07 & 0.80 & 47.04 & -0.73\\
Hispanic & 0.07 & 0.26 & 0.07 & 0.94 & 1.47 & -0.02\\
Age & 33.07 & 11.04 & 33.2 & 24.63 & 13.37  & 8.6\\
Married & 0.70 & 0.46 & 0.71 & 0.17 & 20.54 & 0.54\\
No degree & 0.30 & 0.46 & 0.30 & 0.73 & 16.27 & -0.43\\
Education & 12.0 & 2.86 & 12.03 & 10.38 & 9.85 & 1.65 \\
1975 Earnings   & 13.51 & 9.31 & 13.65 & 3.1 & 19.63 & 10.6\\
1975 Unemp  & 0.11 & 0.32 & 0.11 & 0.37 & 14.29 & -0.26\\
		\hline 
		\end{tabular}
		\end{table}
    \end{center}

\end{frame}


\begin{frame}{Lab}

\url{https://github.com/Mixtape-Sessions/Causal-Inference-2/tree/main/Lab/Lalonde}

\bigskip

Together let's do questions 1 and 2a-c

\end{frame}





\begin{frame}{Concluding remarks}

\begin{itemize}
\item So we hopefully see a few of the key elements of DiD
	\begin{itemize}
	\item Remember: the DiD equation and ATT equation are distinct concepts and definitions
	\item DiD designs can be implemented with OLS specifications that calculate differences in means
	\item Parallel pre-trends and parallel trends are not the same thing -- the first is testable, the latter is not testable
	\item Event studies are mandatory but pre-trends are smoking guns, but can mislead nonetheless
	\end{itemize}
\item Including \emph{time-varying} covariates in the canonical OLS specification requires additional assumptions
\item Doubly robust and IPW incorporate covariates through propensity scores and outcome regressions (or both) using baseline covariate means only
\end{itemize}

\end{frame}


\end{document}
